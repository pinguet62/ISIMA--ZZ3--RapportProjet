\cleardoublepage



\chapter*{Conclusion}
\addcontentsline{toc}{chapter}{Conclusion}

Ce stage avait pour objectif le développement d'une solution de virtualisation, permettant l'utilisation d'outils de manière simple et transparente pour l'utilisateur.
De plus, la maintenance de cette solution devait être la plus efficace possible.

Ma première solution répond à toutes les attentes, en proposant une interface graphique simple d'utilisation, permettant la configuration et le contrôle de la machine virtuelle, ainsi que l'exécution des outils disponibles.
Il est possible d'ajouter de nombreux outils dans la solution de virtualisation, grâce à leur installation sur des disques distincts, ce qui rendra le déploiement optimal.

La solution pourra être améliorée, notamment en ce qui concerne ses performances pour offrir le plus de fluidité à l'utilisateur.
Une étude sur les besoins des utilisateurs permettrait de compléter l'éventail des fonctionnalités.
Grâce aux commentaires du code source, ainsi qu'à la documentation générée, le programme pourra être maintenu très facilement.
\\


Les principaux problèmes rencontrés concernaient la configuration de la machine virtuelle, qui requérait de nombreux réglages, notamment la configuration du réseau dans le cas d'une machine virtuelle minimale.
De plus, les différents script Bash devaient être adaptés pour l'envoi avec Plink, car ils posaient de nombreux problèmes.

Cela m'a donc permis d'approfondir mes connaissances du système d'exploitation Linux, qui possède un fonctionnement bien différent de Windows, mais aussi en réseau, qui était un point clé du projet.
