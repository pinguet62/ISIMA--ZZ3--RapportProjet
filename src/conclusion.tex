\cleardoublepage

\chapter*{Conclusion}
\addcontentsline{toc}{chapter}{Conclusion}


Ce projet avait pour objectif de trouver et développer une solution permettant aux utilisateurs d'envoyer des SMS depuis un ordinateur.
De plus notre cahier des charge imposait une gestion fine des contacts pour obtenir une solution simple et pratique.

Notre première solution remplie entièrement le cahier des charges en proposant une première solution fonctionnelle à l'utilisateur. 
Du coté des applications mobiles qui font office d'intermédiaire entre le site web et le correspondant, le bilan est mitigé. L'application mobile Android est terminée et maintenable grâce à son architecture, alors que l'application iOS est bloquée pour des raisons de sécurités.
Néanmoins, aucune des deux applications n'est aujourd'hui potentiellement publiables sur les markets officiels de Apple et Google. Les techniques employés pour leur fonctionnement les rendes incompatible avec les politiques mises en place par les responsable des systèmes d'exploitations.
Le site web quant à lui propose une première version fonctionnelle pour l'utilisateur. L'utilisateur pourra alors se connecter sur le site web avec
ses identifiants Google, verra ses contacts s'afficher et pourra discuter avec chacun d'entre eux à condition d'avoir installé et activé l'application
mobile iOS/Android sur son téléphone portable.
Le produit est donc entièrement opérationnelle est remplie les conditions de départs correctement.

De nombreuses améliorations devront être effectuées pour permettre de proposer notre solution au grand public.
Néanmoins, avant tout éventuel développement de nouvelle fonctionnalité, il apparaît primordial de travailler sur la couverture de test. Celle-ci 
n'ayant pas fait l'objet d'un travail de notre part, l'évolution du projet pourrait souffrir de cette lacune et perdre en qualité.

L'ajout de fonctionnalités dans notre projet permettra d'offrir aux utilisateurs une solution plus complète qui satisfera les besoins de tous.
Il apparaît alors intéressant d'envisager avoir une première version iOS entièrement fonctionnelle.

De même, nous pensons qu'il serait utile de permettre l'envoi et la réception de MMS ainsi que la possibilité de communiquer vers plusieurs destinataires
en même temps. Il serait aussi utile de proposer une solution sécurisé de sauvegarde des conversations afin que l'utilisateur puisse, lors de la connection vers le site, retrouver ses précédents messages. 