\cleardoublepage



\chapter*{Résumé}

\thispagestyle{empty}



Plus les années passent, plus on on observe que le temps et le confort sont parmi
les plus importantes ressources que nous souhaitions obtenir. La réalisation de ce
projet essaye de subvenir à ces besoins en rendant la vie de ses utilisateurs plus agréable.

Ce travail consiste ainsi à créer un moyen de communication basé sur deux concepts.
Nous voulons profiter de la disponibilité et la présence d'un correspondant sur son 
téléphone ainsi que de la praticité qu'apportent nos ordinateurs pour converser.

Alors comment réunir ces conditions afin de pouvoir communiquer plus rapidement et plus confortablement ?
Nous avons donc choisi de rendre possible l'écriture de SMS depuis un ordinateur.
\\


Pour atteindre cet objectif, la réalisation du projet s'est déroulée en plusieurs étapes.
Nous avons tout d'abord étudié les protocoles de communications que nous souhaitions utiliser
et mis en relation notre étude avec nos contraintes de réalisation. Suite à cela, nous avons
mis en place un schéma global de fonctionnement de notre projet qui nous a permis d'estimer
chacun des modules composant notre travail. Enfin, il a fallu établir deux systèmes de communication.
Tout d'abord entre le téléphone mobile et une entité persistante sur internet puis entre
cette même entité et l'utilisateur. Cette étude terminée, nous avons alors commencé notre
travail d'écriture et tenté de rendre le fonctionnement du projet possible sur plusieurs
types de plateforme mobile (Android et iPhone).
\\


Ce projet a donc été effectué dans les temps et est aujourd'hui entièrement fonctionnel.
Les résultats ont été concluants et nous avons un système capable d'envoyer et de recevoir
des SMS depuis un site internet. Néanmoins, le fonctionnement de ce projet nécessite quelque 
modifications du téléphone utilisé. En effet il nous est apparu impossible de supplanter
des contraintes imposées par les diverses systèmes d'exploitation sur lesquels nous avons travaillé.

Malgré la mise en place d'outils de travail par les constructeurs de téléphone portable,
on a pu observer de nettes différences de politique. En effet, selon la plateforme, les contraintes
seront beaucoup plus importantes et empêcheront le développeur de sortir des sentiers battus. 
Les téléphones fonctionnant sous iOS démontrent clairement cette état d'esprit puisqu'il 
nous a été très difficile d'outrepasser les contraintes afin de réaliser notre projet sur cette plateforme.
\\



\textbf{Mots clé : } SMS, XMPP, Android, iOS, Play Framework
