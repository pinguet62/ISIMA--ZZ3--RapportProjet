\cleardoublepage



\chapter*{Abstract}

\thispagestyle{empty}


Abstract

%%%%%%%%%%%%%%%%%%%%%%%%%%%%%%
%%%%%% Version française %%%%%
%%%%%%%%%%%%%%%%%%%%%%%%%%%%%%

Plus les années passent, plus on on observe que le temps et le confort sont parmi
les plus importantes ressources que nous souhaitions obtenir. La réalisation de ce
projet essaye de subvenir à ces besoins en rendant la vie de ses utilisateurs plus agréable.
\\
Ce travail consiste ainsi à créer un moyen de communication basé sur deux concepts.
Nous voulons profiter de la disponibilité et la présence d'un correspondant sur son 
téléphone ainsi que de la praticité qu'apportent nos ordinateurs pour converser.
\\
Alors comment réunir ces conditions afin de pouvoir communiquer plus rapidement et plus confortablement ?
Nous avons donc choisi de rendre possible l'écriture de SMS depuis un ordinateur. 
\\
Pour atteindre cet objectif, la réalisation du projet s'est déroulé en plusieurs étapes.
Nous avons tout d'abord étudié les protocoles de communications que nous souhaitions utilisé
et mis en relation notre étude avec nos contraintes de réalisation. Suite à cela, nous avons
mis en place un schéma global de fonctionnement de notre projet qui nous a permis d'estimer
chacun des modules composant notre travail. Enfin, il a fallu établir deux systèmes de communications.
Tout d'abord entre le téléphone mobile et une entité persistante sur internet puis entre
cette même entité et l'utilisateur. Cette étude terminé, nous avons alors commencé notre
travail d'écriture et tenté de rendre le fonctionnement du projet possible sur plusieurs
type de plateforme mobile (Android et iPhone).
\\
Ce projet a donc été effectué dans les temps et est aujourd'hui entièrement fonctionnel.
Les résultats ont été concluant et nous avons un système capable d'envoyer et de recevoir
des SMS depuis un site internet. Néanmoins, le fonctionnement de ce projet nécessite quelque 
modifications du téléphone utilisé. En effet il nous est apparu impossible de supplanter
des contraintes imposés par les diverses systèmes d'exploitation sur lesquels nous avons travaillé.
\\
Malgré la mise en place d'outils de travail par les constructeurs de téléphone portable,
on a pu observer de nette différences de politique. En effet, selon la plateforme, les contraintes
seront beaucoup plus importantes et empêche le développeur de sortir des sentiers battus. 
Les téléphones fonctionnant sous iOS démontrent clairement cette état d'esprit puisqu'il 
nous a été très difficile  d'outrepasser les contraintes afin de réaliser notre projet sur cette plateforme.

\newpage


%%%%%%%%%%%%%%%%%%%%%%%%%%%%%%
%%%%%% Version anglaise  %%%%%
%%%%%%%%%%%%%%%%%%%%%%%%%%%%%%

Years after years we observe that time and comfort are among the most important resources
we are willing to obtain. For us, doing this project was all about gaining these precious
resources and making our project customers life more comfortable.
\\
This work consists in using a new way of communicating while using these two previous concept. 
Indeed we are willing to take advantage of the cellphones high availability with the convenience 
of computer communication.
\\
So how could we meet these two conditions in order to offer a comfortable and fast way of communication ?
We simply chose to create a software allowing us to send and receive SMS from a computer.
\\
The project realisation has been done in few steps. First, we have been studying many communication protocol
we wanted to use. With this analysis, we were able to choose the technologies which could overcome our main constraints.
After that, we have been designing a global scheme representating our project allowing us to estimate
each distinct part of the work to  do. Finally, we had to established two systems of communication.
The first one permits the message to be shared between the cellphone and whatsoever entity running
on the internet. The second one forwards the same message from the entity to the user. This study
over, we actually started to write programs and try to make this project possible on multiple
mobile platform (Android and iOS).
\\
This project have been done in time and is now entirely functional. Results are conclusive,
we indeed have a project that allows us to send SMS from a website. Nonetheless, to be able 
to run correctly the software, we need to modify the behaviour of the cellphone used. In fact,
it appears that it was impossible to overcome some of the constraints imposed by the various
cellphone operating systems we worked on. 
\\
Even though, the system usually provides several tools to build applications, we notified huge
politicy difference beetween the providers. Depending on the plateform, constraints will indeed
become more important and prevent the developer from thinking outside the box.
\\
Cellphone using iOS are clearly representating this policy. For our project, despite our success in
making this project a reality, we really had difficult time trying to bypass the unique iOS way of doing things.
\\



\textbf{Keywords : }
SMS, XMPP, Androïd, iOS, Play Framework
