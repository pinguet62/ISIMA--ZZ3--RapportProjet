Androïd est un système d'exploitation open source crée principalement pour les appareils mobiles et 
les tablettes tactiles. Le système d'exploitation basé sur Linux est dirigé par l'entreprise Google
depuis 2005. Il possède une sur-couche utilisant le langage Java sur lequel tourne les applications 
Androïd.
Aujourd'hui, Androïd est l'OS le plus utilisé dans le monde sur les téléphones portables.
\\



%%%%%%%%%%%%%%%%%%%%%%%%%%%%%%%%%%%%%%%%%%%%%%%%%%%%%%%%%%%%%%%%%%%%%%%%%%%%%%%%%%%%%%%%%%%%%%%%%%%%
\subsection{Fonctionnement global}
%%%%%%%%%%%%%%%%%%%%%%%%%%%%%%%%%%%%%%%%%%%%%%%%%%%%%%%%%%%%%%%%%%%%%%%%%%%%%%%%%%%%%%%%%%%%%%%%%%%%

Le fonctionnement global du projet sur la plateforme Androïd est relativement simple. Nous souhaitons
simplement pouvoir utiliser le téléphone portable comme un intermédiaire entre notre site web et le 
correspondant avec qui nous discutons. Pour cela le téléphone fait office de proxy entre les deux 
entités éloignés. Un proxy est un composant qui se place entre un interlocuteur et sont auditeur. Le 
proxy sert principalement à relayer l'information d'un élément vers un autre, il sert d'intermédiaire.
Dans notre cas, le proxy s'occupe de recevoir des SMS et de les rediriger vers le site web. 
Réciproquement, il récupère les messages envoyés par le site web et envoie un SMS vers le correspondant
définis dans le message reçu.
\\



%%%%%%%%%%%%%%%%%%%%%%%%%%%%%%%%%%%%%%%%%%%%%%%%%%%%%%%%%%%%%%%%%%%%%%%%%%%%%%%%%%%%%%%%%%%%%%%%%%%%
\subsection{Création du service}
%%%%%%%%%%%%%%%%%%%%%%%%%%%%%%%%%%%%%%%%%%%%%%%%%%%%%%%%%%%%%%%%%%%%%%%%%%%%%%%%%%%%%%%%%%%%%%%%%%%%

Les applications fonctionnant sous Androïd régissent par un principe simple, celui des "activités".
Pour simplifier, une activité représente une fonctionnalité graphique d'une application. Par exemple,
lorsqu'une application est ouverte, on tombe sur un menu avec les différentes possibilités d'actions 
qui nous sont offertes. Le menu représente donc ici une activité.
\\


Il faut savoir que une activité lorsqu'elle est active occupe le système qui attend des évènements.
Si l'utilisateur ne fait rien alors que le menu est affiché, le système peut s'occuper d'autres 
taches annexes. Si en revanche l'utilisateur navigue dans le menu, alors l'activité est considérée
comme en fonctionnement et bloque toute autres opérations non relative à l'application exécutée.
\\


Dans le cadre de notre application, nous devons à son lancement effectuer une série d'opérations 
afin de mettre l'application en fonctionnement. Cela consiste notamment à, vérifier l'état de la 
connectivité internet, récupérer les identifiants et se connecter sur le service GTalk. Ces 
opérations rende le système bloquant durant leur exécutions. Cela n'est pas un problème lorsque la 
connection s'effectue sans erreurs. Le traitement durant moins d'une seconde, le système n'est pas
gravement impacté. En revanche, si une erreur intervient, le programme va alors essayer de la résoudre.
Cela n'est pas concevable. En effet, une perte soudaine de réseau pourrait par exemple mettre 
l'application dans un état de recherche ce qui est potentiellement très lent. Cette situation rendrait
le terminal complètement blocké durant la recherche. 

Ce comportement n'étant pas souhaitable, nous avons du trouver une solution pour pouvoir outrepasser ce blocage.
\\


Nous avons alors utilisé le concept Androïd de "services". Celui-ci consiste à exécuter une fonctionnalité 
de manière asynchrone. Le service n'impacte pas le système, au contraire il fonctionne avec lui. 

\begin{figure}[!h]
	\center
	\includegraphics[width=13cm]{img/fonctionnement-des-services-android.png}
	\caption{Fonctionnement d'un service Androïd}
	\label{fonctionnement-des-services-android}
\end{figure}

Comme montré dans la figure \ref{fonctionnement-des-services-android}, le service n'est pas  bloquant. 
Lors de son éxecution, le système continue de fonctionner sur d'autres taches en paire avec le service lancé.
\\

Dans notre cas, le service sert à deux taches principales. Tout d'abord il va configurer l'authentification
de l'utilisateur sur GTalk. Ensuite il va simplement se mettre en attente d'évènement et le cas échéant, 
traiter ce meme évènement.
\\



%%%%%%%%%%%%%%%%%%%%%%%%%%%%%%%%%%%%%%%%%%%%%%%%%%%%%%%%%%%%%%%%%%%%%%%%%%%%%%%%%%%%%%%%%%%%%%%%%%%%
\subsection{Authentification sur GTalk}
%%%%%%%%%%%%%%%%%%%%%%%%%%%%%%%%%%%%%%%%%%%%%%%%%%%%%%%%%%%%%%%%%%%%%%%%%%%%%%%%%%%%%%%%%%%%%%%%%%%%

Pour pouvoir relayer communiquer avec le site web, notre projet utilise le protocole XMPP. Plus 
précisément, nous utilisons un même et unique compte GTalk. 

Initialement, le projet devait être conçu pour Androïd. Or chaque utilisateur Androïd a un compte 
Gmail. Ce même compte peut être utiliser pour accéder à tous les services de Google dont GTalk.
C'est une des raisons qui a fait que nous avons choisi XMPP plutot qu'un autre protocole. L'utilisateur
n'a ainsi pas besoin de créer quelconque nouveaux comptes et peux utiliser celui qu'il possède déjà.
\\


Pour pouvoir converser avec le site web, l'application a besoin de s'authentifier auprès du service
GTalk. Pour cela, nous sommes passés par deux étapes. 

Initialement, nous utilisions la méthode classique d'authentification. Lorsque l'utilisateur lance 
l'application, celle-ci lui demande de rentrer ses identifiant et mot de passe. Cela effectué, 
l'application se charge de se connecter et de s'authentifier sur les serveurs GTalk en utilisant les
paramètres rentrés par l'utilisateur.

Cette solution qui est la plus simple à implémenter mais n'est malheureusement pas la plus agréable
à utiliser ni la plus sure. Nous forçons en effet l'utilisateur à rentrer ses identifiants ce qui 
pourrait paraitre légitime mais reste contraignant pour l'utilisateur. Celui-ci ne sait pas ce que 
nous faisons avec son mot de passe. Une application malveillante pourrait récupérer les identifiants
et s'en servir à des fins illégales. Dans notre cas, même si nous n'avions aucune mauvaise intentions,
nous ne pouvons demander aux utilisateurs de nous faire confiance. Nous avons donc opté pour une 
deuxième solution, l'identification avec OAuth2.0.

Comme vu précédemment, OAuth permet de réaliser des actions en agissant à la place de l'utilisateur. 
Ici nous souhaitions simplement utiliser le compte GTalk de l'utilisateur sans que celui-ci ai à rentrer
ses identifiants. 

Le fonctionnement est simple. Lorsque utilisateur lance l'application, celle-ci demande tout d'abord
à l'utilisateur quel compte Gmail il veut utiliser parmi ceux présent sur sont téléphone. Dans le cas
ou il n'existe qu'un seul compte, celui-ci est choisi par défaut.
\\


Suite à cela, le protocole OAuth va demander à l'utilisateur une seule et unique fois si il autorise 
l'application à accéder à son compte GTalk. La validation effectué, l'application pourra se connecter 
à chaque lancement, automatiquement et sans importuner l'utilisateur. 
\\


Contrairement a l'authentification sur le site web, l'authentification sur une plateforme Androïd est
théoriquement plus simple. En effet, une fois que l'utilisateur a donné son accord, le serveur 
d'authentification de Google va directement envoyer le token d'authentification. Il n'y a alors pas 
besoin d'effectuer une deuxième étape pour récupérer ce dernier comme le montre le diagramme 
\ref{obtention-token-avec-android}.

%%%%%%%%%%%%%%%%%%%%%%%%%%%%
%title Envoi d'un SMS

%Utilisateur->Application: dégation de l'authentification
%Application->Serveur Google: requete pour un token d'authentification
%Serveur Google->Application: token d'authentification
%Application->serveur GTalk: authentification(token)
%serveur GTalk->Application: authentification validé
%%%%%%%%%%%%%%%%%%%%%%%%%%%%

\begin{figure}[!h]
	\center
	\includegraphics[width=15cm]{img/obtention-token-avec-android.png}
	\caption{Procédure d'obtention d'un token avec un appareil Androïd}
	\label{obtention-token-avec-android}
\end{figure}

Néanmoins, une des difficultés d'implémenter ce protocole est que la libraire de gestion de compte
XMPP pour Androïd dénommé Asmack est différente de celle utiliser pour le site web. Asmack est un fork 
de la librairie Smack pour Androïd et est l'unique librairie gérant les comptes XMPP présente sur
Androïd. Ne souhaitant pas ré-implémenter toute les fonctionnalités présentes dans celle-ci, nous avons
choisi de l'utiliser. Malheureusement, les différences entre les deux librairies nous ont empêché de 
ré-implémenter notre précédent travail effectué pour le site web qui était fonctionnel. 

Le langage Java fonctionne sur une machine virtuelle. Celle-ci contient des fonctionnalités basique sur
lesquelles le langage peut s'appuyer. Dalvik, la machine virtuelle pour Java présente sur Androïd n'est 
pas exactement la même que la machine virtuelle basique. En effet, pour des soucis de rapidité et de 
légèreté, certaines parties de Dalvik ont été réécrites pour en faire une machine virtuelle plus adapté
aux plateformes mobiles.

Smack a été développé sur la machine virtuelle Java basique tandis que Asmack a été développé sur Dalvik.
Cette différence implique des changements quand aux fonctionnements de celles-ci. 

Finalement, en ré-implémentant certaines fonctionnalités, nous avons réussi à nous authentifier avec OAuth.
\\


Authentifier un utilisateur via OAuth2.0 sur Androïd a était un réel challenge. Il nous a fallu comprendre 
et trouver des équivalences au fonctionnement de Smack. 
\\



%%%%%%%%%%%%%%%%%%%%%%%%%%%%%%%%%%%%%%%%%%%%%%%%%%%%%%%%%%%%%%%%%%%%%%%%%%%%%%%%%%%%%%%%%%%%%%%%%%%%
\subsection{Envoi et réception d'un SMS}
%%%%%%%%%%%%%%%%%%%%%%%%%%%%%%%%%%%%%%%%%%%%%%%%%%%%%%%%%%%%%%%%%%%%%%%%%%%%%%%%%%%%%%%%%%%%%%%%%%%%

\subsubsection{Réception d'un SMS}

Une fois authentifié, nous souhaitons pouvoir utiliser notre téléphone comme proxy. Notre application
doit en effet servir d'intermédiaire entre le site web et un correspondant. Pour cela elle doit tout
d'abord surveiller l'arrivée de nouveaux messages. 

Sur Androïd, le monitoring d'évènements n'implique pas de devoirs sortir des sentiers battus. En effet,
des outils mis en place par Google autorise la surveillance du comportement du téléphone. Dans notre cas 
nous avons pu utiliser le principe de "Broadcast Receiver" du système. 

Un broadcast receiver est un outils qui permet de rendre une application consciente de son environnement.
Cela permet généralement à l'application qui l'utilise d'etre prévenu lors de nouvelles notifications ou 
d'un comportement particulier du téléphone.
Son fonctionnement est simple. Il permet de s'enregistrer auprès du téléphone qui maintient une base de donnée des 
membres à avertir lors d'un changement particulier.

\begin{figure}[!h]
	\center
	\includegraphics[width=12cm]{img/broadcast-receivers.png}
	\caption{Fonctionnement des broadcasts receivers sur Androïd}
	\label{broadcast-receivers}
\end{figure}

Le schéma \ref{broadcast-receivers} permet de voir le système du téléphone qui contient les différents broadcast receiver à avertir en cas de nouveaux messages. Ici l'application de rédaction de SMS ainsi que deux autres 
applications seront avertis.

Dans notre cas, nous nous en sommes servis pour d'une part être avertis de l'arrivée de nouveaux SMS et d'autre part pour pouvoir récupérer le dit message et le traiter.
\\


\subsubsection{Envoi d'un SMS}

L'envoi quand à lui est beaucoup plus trivial. Androïd met à disposition un outils permettant d'envoyer
des messages.
\\



%%%%%%%%%%%%%%%%%%%%%%%%%%%%%%%%%%%%%%%%%%%%%%%%%%%%%%%%%%%%%%%%%%%%%%%%%%%%%%%%%%%%%%%%%%%%%%%%%%%%
\subsection{Traitement des évènements}
%%%%%%%%%%%%%%%%%%%%%%%%%%%%%%%%%%%%%%%%%%%%%%%%%%%%%%%%%%%%%%%%%%%%%%%%%%%%%%%%%%%%%%%%%%%%%%%%%%%%

\subsubsection{Architecture de l'application}

Pour pouvoir réagir en fonction d'un évènement, notre application a successivement adopté deux méthodes.

La première a été réalisé pour répondre aux besoins de notre projet. Il s'agissait d'un test sur le 
type de l'évènement entrant. Plus précisément, il s'agissait de vérifier si l'évènement en question 
était la réception d'un nouveaux SMS ou bien la réception d'un nouveau message XMPP en provenance du 
site web. En fonction du type de message en entrée, l'application décidé d'envoyer transmettre le message
vers le bon destinataire. 

Ce fonctionnement basique convenait totalement à nos besoins mais par soucis de rendre notre application
évolutive nous avons travaillé sur une deuxième solutions.

Lors de la réception d'un message XMPP, nous souhaitions pouvoir réagir différemment en fonction du contenu
du message. Dans le cadre de notre projet, nous n'avions comme but que l'envoi de sms à partir du webservice
seulement. Notre travail respecté donc ce principe mais ne permettait aucune évolutivité.

A supposer qu'un tiers soit intéressé par la communication entre le webservice et le téléphone par 
l'intermédiaire de XMPP mais que son but ne soit pas d'envoyer de simple sms, nous avons implémenté une
architecture permettant de rajouter facilement de nouvelles fonctionnalités. Pour donner des exemples, 
un utilisateur pourrait décider d'implémenter une fonctionnalité lui permettant de déclencher la capture
de photos de son téléphone à partir du webserver.

Pour permettre cette évolutivité, nous avons réalisé une architecture qui s'adapte automatiquement en
fonction des messages reçus. Il s'agit ici des patrons de conceptions fabrique et stratégie. 
\\


Le patron de stratégie est un patron de conception de type comportemental. Il permet de d'adopter 
dynamiquement un comportement particulier en fonction des conditions de son exécution. Le schéma 
\ref{pattern_strategie} décrit le fonctionnement du modèle.

\begin{figure}[H]
	\center
	\includegraphics[width=12cm]{img/pattern_strategie.png}
	\caption{Diagramme UML du pattern stratégie}
	\label{pattern_strategie}
\end{figure}

Dans notre cas, il permet de créer différents type de stratégie qui pourront potentiellement utiliser. 
Notre projet ne contenait que la stratégie "envoi de SMS" mais, avec cette architecture, il est possible 
de rajouter d'autres stratégies comme l'extinction du téléphone ou encore l'envoi d'un email par exemple.
\\


La fabrique décrit sur le schéma \ref{fabrique} quand à elle permet de créer des objets sans en connaitre l'éxistence avant le lancement du programme.

\begin{figure}[H]
	\center
	\includegraphics[width=10cm]{img/fabrique.png}
	\caption{Diagramme UML du pattern fabrique}
	\label{fabrique}
\end{figure}

Dans notre cas, elle permet d'utiliser de créer les stratégies et de les utiliser. En effet, la fabrique 
qui va analyser chaque nouveau message. En fonction de leurs contenus, elle devra décider d'une stratégie
et l'appliquer.
\\

La figure \ref{fonctionnement-strategie-factorie} montre le fonctionnement du projet lors de l'arrivée d'un
nouveau message. L'intérêt principal de ce pattern est l'évolutivité qu'il apporte. Le choix dynamique à
l'exécution de la stratégie à un intérêt pour le développeur. En effet, celui-ci n'a plus besoin de se
soucier de l'implémentation du choix de la stratégie. Cette partie est figé et est fonctionnelle en plus d'être très facilement complété.

\begin{figure}[H]
	\center
	\includegraphics[width=10cm]{img/fonctionnement-strategie-factorie.png}
	\caption{Diagramme UML du pattern fabrique}
	\label{fonctionnement-strategie-factorie}
\end{figure}


\subsubsection{Traitements des évènements}

Selon l'évènement reçu, le traitement à effectuer n'est pas le même. Pour notre application, nous avons 
créer recensés deux évènements et traitements associés possible.

La réception d'un SMS est notre premier cas. Comme le montre le schéma \ref{encapsulation-sms}, le traitement se 
décompose en trois étapes. 

\begin{figure}[!h]
	\center
	\includegraphics[width=12cm]{img/encapsulation-sms.png}
	\caption{Traitements lors de la réception d'un SMS}
	\label{encapsulation-sms}
\end{figure}

Il faut tout d'abord récupérer le contenu du message ainsi que son auteur. Puis il fallait encapsuler 
ses données dans un message au format Json. Enfin nous terminions par envoyer le message encapsulé vers 
le webservice qui s'occupera de notifier l'utilisateur de l'arrivée d'un nouveau message.

Concernant le deuxième évènement possible, il s'agit de la réception d'un nouveau message XMPP. Comme le
montre le schéma \ref{desencapsulation}, contrairement au premier traitement, il va être désencapsulé pour 
pouvoir y récupérer l'information utile. Cela fait, notre fabrique va s'occuper de d'appliquer la stratégie 
adapter pour ce nouveau message. Dans notre cas, puisque nous avions juste la stratégie d'envoi de SMS, c'était toujours celle-ci qui était appliqué.

%XMPP->Application: arrivé d'un nouveau message XMPP
%Application->Application: désencapsulation du Json contenu dans le message
%Application->SMS: envoie d'un SMS

\begin{figure}[!h]
	\center
	\includegraphics[width=12cm]{img/desencapsulation.png}
	\caption{Traitements lors de la réception d'un message XMPP}
	\label{desencapsulation}
\end{figure}
