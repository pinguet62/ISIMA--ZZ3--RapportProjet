\cleardoublepage



\chapter{Réalisation de la solution}

\paragraph{}
Pour ce projet, nous avions initialement établis de pouvoir le faire fonctionner sur plusieurs 
architectures mobiles. Ayant des cours sur les deux plateformes phares actuelles, nous avons choisi
d'étudier les mobiles de types iOS et Android. De plus comme nous possédions chacun un téléphone 
sur les deux systèmes d'exploitation, il nous est apparu intelligent de se répartir le travail en 
travaillant chacun sur un OS.
\\
La partie qui suit traite donc de l'étude et de la réalisation du projet sur l'environnement Android.


%%%%%%%%%%%%%%%%%%%%%%%%%%%%%%%%%%%%%%%%%%%%%%%%%%%%%%%%%%%%%%%%%%%%%%%%%%%%%%%%%%%%%%%%%%%%%%%%%%%%
%%%%%%%%%%%%%%%%%%%%%%%%%%%%%%%%%%%%%%%%%%%%%%%%%%%%%%%%%%%%%%%%%%%%%%%%%%%%%%%%%%%%%%%%%%%%%%%%%%%%
%%%%%%%%%%%%%%%%%%%%%%%%%%%%%%%%%%%%%%%%%%%%%%%%%%%%%%%%%%%%%%%%%%%%%%%%%%%%%%%%%%%%%%%%%%%%%%%%%%%%
%%%%%%%%%%%%%%%%%%%%%%%%%%%%%%%%%%%%%%%%%%%%%%%%%%%%%%%%%%%%%%%%%%%%%%%%%%%%%%%%%%%%%%%%%%%%%%%%%%%%

\section{Android}

%%%%%%%%%%%%%%%%%%%%%%%%%%%%%%%%%%%%%%%%%%%%%%%%%%%%%%%%%%%%%%%%%%%%%%%%%%%%%%%%%%%%%%%%%%%%%%%%%%%%
%%%%%%%%%%%%%%%%%%%%%%%%%%%%%%%%%%%%%%%%%%%%%%%%%%%%%%%%%%%%%%%%%%%%%%%%%%%%%%%%%%%%%%%%%%%%%%%%%%%%
%%%%%%%%%%%%%%%%%%%%%%%%%%%%%%%%%%%%%%%%%%%%%%%%%%%%%%%%%%%%%%%%%%%%%%%%%%%%%%%%%%%%%%%%%%%%%%%%%%%%


\paragraph{}
Android est un système d'exploitation open source crée principalement pour les appareils mobiles et 
les tablettes tactiles. Le système d'exploitation basé sur Linux est dirigé par l'entreprise Google
depuis 2005. Il possède une surcouche utilisant le langage Java sur lequel tourne les applications 
Android.
\\
Aujourd'hui, Android est l'OS le plus utilisé dans le monde sur les téléphones portables.


%%%%%%%%%%%%%%%%%%%%%%%%%%%%%%%%%%%%%%%%%%%%%%%%%%%%%%%%%%%%%%%%%%%%%%%%%%%%%%%%%%%%%%%%%%%%%%%%%%%%
\subsection{Fonctionnement global}
%%%%%%%%%%%%%%%%%%%%%%%%%%%%%%%%%%%%%%%%%%%%%%%%%%%%%%%%%%%%%%%%%%%%%%%%%%%%%%%%%%%%%%%%%%%%%%%%%%%%


\paragraph{}
Le fonctionnement global du projet sur la plateforme Android est relativement simple. Nous souhaitons
simplement pouvoir utiliser le téléphone portable comme un intermédiaire entre notre site web et le 
correspondant avec qui nous discutons. Pour cela le téléphone fait office de proxy entre les deux 
entités éloignés. Un proxy est un composant qui se place entre un interlocuteur et sont auditeur. Le 
proxy sert principalement à relayer l'information d'un élément vers un autre, il sert d'intermédiaire.
Dans notre cas, le proxy s'occuppe de recevoir des SMS et de les rediriger vers le site web. 
Réciproquement, il récupère les messages envoyés par le site web et envoie un SMS vers le correspondant
définis dans le message reçu.




%%%%%%%%%%%%%%%%%%%%%%%%%%%%%%%%%%%%%%%%%%%%%%%%%%%%%%%%%%%%%%%%%%%%%%%%%%%%%%%%%%%%%%%%%%%%%%%%%%%%
\subsection{Création du service}
%%%%%%%%%%%%%%%%%%%%%%%%%%%%%%%%%%%%%%%%%%%%%%%%%%%%%%%%%%%%%%%%%%%%%%%%%%%%%%%%%%%%%%%%%%%%%%%%%%%%



\paragraph{}
Les applications fonctionnant sous Android régissent par un principe simple, celui des "activités".
Pour simplifier, une activité représente une fonctionnalité graphique d'une application. Par exemple,
lorsqu'une application est ouverte, on tombe sur un menu avec les différentes possibilités d'actions 
qui nous sont offertes. Le menu représente donc ici une activité. 
\\
Il faut savoir que une activité lorsqu'elle est active occupe le système qui attend des évènements.
Si l'utilisateur ne fait rien alors que le menu est affiché, le système peut s'occupper d'autres 
taches annexes. Si en revanche l'utilisateur navigue dans le menu, alors l'activité est considérée
comme en fonctionnement et bloque toute autres opérations non relative à l'application executée.
\\
Dans le cadre de notre application, nous devons à son lancement effectuer une série d'opérations 
afin de mettre l'application en fonctionnement. Cela consiste notamment à, vérifier l'état de la 
connectivité internet, récupérer les identifiants et se connecter sur le service GTalk. Ces 
opérations rende le système bloquant durant leur exécutions. Cela n'est pas un problème lorsque la 
connection s'effectue sans erreurs. Le traitement durant moins d'une seconde, le système n'est pas
gravement impacté. En revanche, si une erreur intervient, le programme va alors essayer de la résoudre.
Cela n'est pas concevable. En effet, une perte soudaine de réseau pourrait par exemple mettre 
l'application dans un état de recherche ce qui est potentiellement très lent. Cette situation rendrait
le terminal complètement blocké durant la recherche. 
\\
Ce comportement n'étant pas souhaitable, nous avons du trouver une solution pour pouvoir outrepasser ce blocage.
\\
Nous avons alors utilisé le concept Android de "services". Celui-ci consiste à executer une fonctionnalité 
de manière asynchrone. Le service n'impacte pas le système, au contraire il fonctionne avec lui. 

\\
<img service fonctionnement>
\\

Comme montré ci dessus, le service n'est pas bloquant. Lors de son éxecution, le système continue de 
fonctionner sur d'autres taches en paire avec le service lancé.

\paragraph{}
Dans notre cas, le service sert à deux taches principales. Tout d'abord il va configurer l'authentification
de l'utilisateur sur GTalk. Ensuite il va simplement se mettre en attente d'évènement et le cas échéant, 
traiter ce meme évènement.




%%%%%%%%%%%%%%%%%%%%%%%%%%%%%%%%%%%%%%%%%%%%%%%%%%%%%%%%%%%%%%%%%%%%%%%%%%%%%%%%%%%%%%%%%%%%%%%%%%%%
\subsection{Authentification sur GTalk}
%%%%%%%%%%%%%%%%%%%%%%%%%%%%%%%%%%%%%%%%%%%%%%%%%%%%%%%%%%%%%%%%%%%%%%%%%%%%%%%%%%%%%%%%%%%%%%%%%%%%


%%%%%%%%%%%%%%%%%%%%%%%%%%%%%%%%%%%%%%%%%%%%%%%%%%%%%%%%%%%%%%%%%%%%%%%%%%%%%%%%%%%%%%%%%%%%%%%%%%%%
\subsection{Reception d'un SMS}
%%%%%%%%%%%%%%%%%%%%%%%%%%%%%%%%%%%%%%%%%%%%%%%%%%%%%%%%%%%%%%%%%%%%%%%%%%%%%%%%%%%%%%%%%%%%%%%%%%%%


%%%%%%%%%%%%%%%%%%%%%%%%%%%%%%%%%%%%%%%%%%%%%%%%%%%%%%%%%%%%%%%%%%%%%%%%%%%%%%%%%%%%%%%%%%%%%%%%%%%%
\subsection{Envoi d'un SMS}
%%%%%%%%%%%%%%%%%%%%%%%%%%%%%%%%%%%%%%%%%%%%%%%%%%%%%%%%%%%%%%%%%%%%%%%%%%%%%%%%%%%%%%%%%%%%%%%%%%%%


%%%%%%%%%%%%%%%%%%%%%%%%%%%%%%%%%%%%%%%%%%%%%%%%%%%%%%%%%%%%%%%%%%%%%%%%%%%%%%%%%%%%%%%%%%%%%%%%%%%%
\subsection{Traitement des évènements}
%%%%%%%%%%%%%%%%%%%%%%%%%%%%%%%%%%%%%%%%%%%%%%%%%%%%%%%%%%%%%%%%%%%%%%%%%%%%%%%%%%%%%%%%%%%%%%%%%%%%
\\


