\cleardoublepage



\chapter*{Résumé}

\thispagestyle{empty}



Mon stage de seconde année d'étude à l'ISIMA avait pour but de développer une solution de virtualisation, permettant d'exécuter des outils scientifiques uniquement disponible sous Linux.
\\


Tout d'abord j'ai construit et configuré une première machine virtuelle Linux complète, sur laquelle se basera ma première solution.
Son environnement de développement complet permettra de compiler des outils scientifiques.

Grâce à l'analyse des différentes techniques existantes et utilisables, j'ai pu m'orienter vers les solutions que j'utiliserai.
Il fallait aussi penser à obtenir la solution la plus pratique pour l'utilisateur mais aussi pour son déploiement dans l'entreprise.

Une fois les programmes développés et les tests effectués, il ne restait plus qu'à améliorer les différents points qui posent problème avant de construire une machine virtuelle minimale qui offrira de meilleures performances.
\\



\textbf{Mots clé : }
Virtualisation, VirtualBox, VBoxManage, Outil scientifique, Python, wxPython